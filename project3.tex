\documentclass[11pt]{article}

\usepackage{amssymb}
\usepackage{amsmath}
\usepackage{graphicx}
\usepackage{hyperref}

\def\N{{\mathbb N}}
\def\NN{{\mathcal N}}
\def\R{{\mathbb R}}
\def\E{{\mathbb E}}
\def\rank{{\mathrm{rank}}}
\def\tr{{\mathrm{trace}}}
\def\P{{\mathrm{Prob}}}
\def\sign{{\mathrm{sign}}}
\def\diag{{\mathrm{diag}}}

\setlength{\oddsidemargin}{0.25 in}
\setlength{\evensidemargin}{-0.25 in}
\setlength{\topmargin}{-0.6 in}
\setlength{\textwidth}{6.5 in}
\setlength{\textheight}{8.5 in}
\setlength{\headsep}{0.75 in}
\setlength{\parindent}{0.25 in}
\setlength{\parskip}{0.1 in}

\newcommand{\lecture}[4]{
   \pagestyle{myheadings}
   \thispagestyle{plain}
   \newpage
   \setcounter{page}{1}
   \setcounter{section}{0}
   \noindent
   \begin{center}
   \framebox{
      \vbox{\vspace{2mm}
    \hbox to 6.28in { {\bf Math 6380p: Adv. Top. Deep Learning \hfill #4} }
       \vspace{6mm}
       \hbox to 6.28in { {\Large \hfill #1  \hfill} }
       \vspace{6mm}
       \hbox to 6.28in { {\it Instructor: #2\hfill #3} }
      \vspace{2mm}}
   }
   \end{center}
   \markboth{#1}{#1}
   \vspace*{4mm}
}


\begin{document}

\lecture{Final Projects.}{Yuan Yao}{Due: 00:00am Monday 17 Dec, 2018}{28 Nov, 2018}

%The problem below marked by $^*$ is optional with bonus credits. % For the experimental problem, include the source codes which are runnable under standard settings. 
%
%\begin{enumerate}
%
%\item {\em Manifold Learning}: The following codes by Todd Wittman contain major manifold learning algorithms talked on class.
%
%\url{http://www.math.pku.edu.cn/teachers/yaoy/Spring2011/matlab/mani.m}
%
%Precisely, eight algorithms are implemented in the codes: MDS, PCA, ISOMAP, LLE, Hessian Eigenmap, Laplacian Eigenmap, Diffusion Map, and LTSA. 
%The following nine examples are given to compare these methods,
%\begin{enumerate}
%\item Swiss roll;
%\item Swiss hole;
%\item Corner Planes;
%\item Punctured Sphere;
%\item Twin Peaks;
%\item 3D Clusters;
%\item Toroidal Helix;
%\item Gaussian;
%\item Occluded Disks.
%\end{enumerate}
%Run the codes for each of the nine examples, and analyze the phenomena you observed. 
%
%\end{enumerate}

%\newpage


\section{Project Requirement}

This project as a warm-up aims to explore feature extractions using existing networks, such as pre-trained deep neural networks and scattering nets, in image classifications with traditional machine learning methods.
\begin{enumerate}
\item Pick up ONE (or more if you like) favourite dataset below to work. If you would like to work on a different problem outside the candidates we proposed, please email course instructor about your proposal.  
\item Team work: we encourage you to form small team, up to FOUR persons per group, to work on the same problem. Each team just submit ONE report, \emph{with a clear remark on each person's contribution}. The report can be in the format of either Python (Jupyter) Notebooks with a detailed documentation (preferred format), a \emph{technical report within 8 pages}, e.g. NIPS conference style 
\begin{center}
\url{https://nips.cc/Conferences/2016/PaperInformation/StyleFiles} 
\end{center}
or of a \emph{poster}, e.g. 
\begin{center}%\url{http://math.stanford.edu/~yuany/publications/poster_CleaveBioCPH2017_ForReview.pptx}
\url{https://github.com/yuany-pku/2017_math6380/blob/master/project1/DongLoXia_poster.pptx}
\end{center}
\item In the report, show your proposed scientific questions to explore and main results with a careful analysis supporting the results toward answering your problems. Remember: scientific analysis and reasoning are more important than merely the performance tables. Separate source codes may be submitted through email as a zip file, GitHub link, or as an appendix if it is not large.    
\item Submit your report by email or paper version no later than the deadline, to the following address (\href{mailto:deeplearning.math@gmail.com}{deeplearning.math@gmail.com}) with Title: \underline{Math 6380P: Project 1}. % (\href{mailto:datascience\_hw@126.com}{datascience\_hw@126.com}). 
\end{enumerate}


\section{New Challenges Self-Proposed by Classmates} 

\subsection{Smartphones glass defects detection and pixel-wise classification}
This project is about smartphones glass defects detection and pixel-wise classification. The purpose is to analyze the response of Scatnet, Resnet, VGG, and DCF-Based Networks in terms of accuracy and speed.

Robotics and Multi-Perception Lab (ram-lab.com) has created a glass defects datasets which have 80-100 images of smartphones glasses with defects such as Scratch, Pit, Crack, Chip, Dirt. Each image is taken from the 16K line camera. Size of each image is about 400-500MBs and the total size is about 21GBs. The main task to detect the defects and pixel-wise classification of these defects. The minimum defect to detect and classify should be about $\sim$10 micrometers.  Each pixel is about $\sim$0.5-micron size.

You can request dataset from (\url{mumbhutta@connect.ust.hk}) if interested. This challenge is proposed and pursued by Muhammad Usman Maqbool BHUTTA (\url{mumbhutta@connect.ust.hk}), Yuan LAN (\url{ylanaa@connect.ust.hk}), and Candi ZHENG (\url{czhengac@connect.ust.hk}).
\begin{figure}[htbp]
\begin{centering}
	\includegraphics[scale=0.2]{glass.png}  
	\caption{A sample of glass defect that contains all the four types, scatches, cracks, chip, and dirt.}
	\label{fig:heatmap}
\end{centering}
\end{figure}


%\subsection{Smartphones glass defects detection and pixel-wise classification}
%This project is about smartphones glass defects detection and pixel-wise classification. The purpose is to analyze the response of Scatnet, Resnet, VGG, and DCF-Based Network in terms of accuracy and speed.
%
%Robotics and Multi-Perception Lab (ram-lab.com) has created a glass defects datasets which have 80-100 images of smartphones glasses. Each image is taken from the 16K line camera. Size of each image is about 400-500MBs and the total size is about 21GBs. The main task to detect the defects and pixel-wise classification of these defects. The minimum defect to detect and classify should be about $\sim$10 micrometers.  Each pixel is about $\sim$0.5-micron size.
%
%You can request dataset from (\url{mumbhutta@connect.ust.hk}) if interested. This challenge is proposed and pursued by Muhammad Usman Maqbool BHUTTA (\url{mumbhutta@connect.ust.hk}), Yuan LAN (\url{ylanaa@connect.ust.hk}), and Candi ZHENG (\url{czhengac@connect.ust.hk}).

\subsection{EmoContext: A Shared Task on Contextual Emotion Detection in Text}
We routinely experience emotions such as happiness, anger, sadness etc. As humans, on reading ``Why don't you ever text me!", we can either interpret it as a sad or an angry emotion in absence of context; and the same ambiguity exists for machines as well. Lackness of facial expressions and voice modulations makes detecting emotions in text a challenging problem. However, as we increasingly communicate using text messaging applications and digital agents, contextual emotion detection in text is gaining importance to provide emotionally aware responses to users. The shared task aims to bring more research to the problem of contextual emotion detection in text. 

In this task, you are given a textual dialogue, i.e. a user utterance along with two turns of context. One has to classify the emotion of user utterance as one of the emotion classes: Happy, Sad, Angry or Others. The training data set contain 15K records for emotion classes, i.e., Happy, Sad and Angry combined, and contains 15K records not belonging to any of the aforementioned emotion classes. This challenge has been proposed as SemEval 2019 task (\url{http://alt.qcri.org/semeval2019/}), there are already more than 100 submission in the scorer system. More information can be found in \url{https://www.humanizing-ai.com/emocontext.html} and some baseline code in \url{https://github.com/SenticNet/conv-emotion/blob/master/bc-LSTM/baseline.py}.

This challenge is proposed and pursued by Andrea MADOTTO (\url{amadotto@connect.ust.hk}), Genta Indra WINATA (\url{giwinata@connect.ust.hk}), Zhaojiang LIN (\url{zlinao@connect.ust.hk}), and Jay SHIN (\url{jay.shin@connect.ust.hk}) from CAiRE. 

\subsection{Exploring the Robustness of Neural Network}
Despite its numerous successful applications, neural network, it is susceptible to natural noises or artificial attacks. Therefore, the robustness of neural network arouse our interests in this challenge.

Two particular tasks will be pursued here. On one hand, we will verify DCFNet's robustness against high-frequency noise. As proposed in the (\href{https://arxiv.org/abs/1802.04145}{https://arxiv.org/abs/1802.04145}), DCFNet's structured filters should equip it the ability against high-frequency noise. Quantitative verification on this hypothesis will be studied. On the other hand, we will extend the term ``noise" to adversarial examples, which is introduced in Lecture 22. In particular, we will try to reproduce the method proposed in (\url{https://arxiv.org/abs/1704.08847}), which try to improve robustness of networks by adding constraint on its Lipschitz constant. The Lipschitz constraint can be done for each layer by adding orthogonal constraint on the weight matrix (e.g. convolutional operators). The singular values of the weight matrix are constrained to be 1 and the Lipschitz constant of whole neural network is smaller than 1. It is hoped that smaller perturbation of input will not affect output too much.

One can use various attack methods to attack a network to see whether it is robust to adversarials. Several Attack method have been introduced in Lecture 22 including FGSM, PGD and CW etc. One can use MNIST or CIFAR10 to do this task with whatever network architecture that one wants to explore like VGG16 or ResNet18.

This challenge is proposed and pursued by Zhichao, HUANG (\url{zhuangbx@connect.ust.hk}) and Zhicong, LIANG(\url{zliangak@connect.ust.hk}).

\subsection{Denoising in Cryo-EM Imaging Problem}
\subsubsection{Introduction}
The Cryo-EM images are the projection images of frozen samples by the electron microscope, whose process may introduce large and irregular noise. In order to classify the different structure of elements and reconstruct the biological structure well, denoising is becoming a very important process. This challenge is proposed and pursued by Hanlin GU (\url{hguaf@connect.ust.hk}) and his peers in Prof. Xuhui HUANG's Lab at HKUST. 

\subsubsection{Data}
The data is mainly a special structure of bacterial RNA Polymerase which includes 50000 clean images of size 128*128 (\url{http://143.89.53.65/shared_folder/hanlin_project/}) (need to \texttt{import mrcfile} in python to read data), you can add any types of noise in different SNR (signal noise ratio) in clean images using python function (or any functions you like):

\texttt{skimage.util.random\_noise(image, mode=, seed=None, clip=True, **kwargs)}

\subsubsection{Process}
Try traditional denoising methods, for example transform domain methods (BM3D), dictionary learning methods (KSVD) and so on. 
Or you can try deep learning methods like VAE or GAN to denoise, the following is one implementation using GAN by Cianfrocco's Lab in the University of Michigan:

\url{https://github.com/cianfrocco-lab/GAN-for-Cryo-EM-image-denoising}\\

\noindent Can you try to create your own new network architecture to denoise Cryo-EM images? After denoising, you can set an criterion to evaluate the performance, a direct method is calculate the mean square error between clean images and noisy denoised images. A good try is to compare different methods in different SNR levels to understand the performance of different denoising methods well.

\subsubsection{Further exploration}
There are several challenges in Cryo-EM denoising: (A) Can the method denoise well when SNR is low? (B) Is your method also suitable for experimental real world data? Some real world experimental data can be downloaded in the website \url{https://www.ebi.ac.uk/pdbe/emdb/empiar/}. For experimental data, it it worth to mention that some references may be derived by softwares: CRYOSPARC \url{https://cryosparc.com/docs/tutorials/}, and you can also using this software to reconstruct 3D structure.


\section{Reinforcement Learning for Image Classification with Recurrent Attention Models} 

This task basically required you to reproduce some key result of Recurrent Attention Models, proposed by Mnih et al. (2014). (See \url{https://arxiv.org/abs/1406.6247})

\subsection{Cluttered MINIST}

As you've known, the original MNIST dataset is a canonical dataset used for illustration of deep learning, where a simple multi-layer perceptron could get a very high accuracy. Therefore the original MNIST is augmented with additional noise and distortion in order to make the problem more challenging and closer towards real-world problems. 

\begin{figure}[h]
\center
\includegraphics[width=0.7\textwidth]{clutter_mnist.png}  
\caption{Cluttered MNIST data sample.} 
\label{fig:cmnist}
\end{figure}

The dataset can be downloaded from \url{https://github.com/deepmind/mnist-cluttered}.

\subsection{Raphael Drawings}
For those who are interested in exploring authentication of Raphael drawings, you are encouraged to use the Raphael dataset:

\url{https://drive.google.com/folderview?id=0B-yDtwSjhaSCZ2FqN3AxQ3NJNTA&usp=sharing}

\noindent that will be discussed in more detail later. 

\subsection{Recurrent Attention Models (RAMs)}

The efficiency of human eyes when looking at images lies in attentions -- human do not process all the input information but instead use attention to select partial and important information for perception. This is particularly important for computer vision when the input images are too big to get fully fed into a convolutional neural networks. To overcome this hurdle, the general idea of RAM is to model human attention €using recurrent neural networks by dynamically restricting on interesting parts of the image where one can get most of information, followed by reinforcement learning with rewards toward minimizing misclassification errors. 

You're suggested to implement the following networks and train the networks on the above Cluttered MINIST dataset.

\begin{figure}[h]
\center
\includegraphics[width=0.7\textwidth]{network_RAM.png}  
\caption{Diagram of RAM.} 
\label{fig:ram}
\end{figure}

\begin{itemize}
	\item Glimpse Sensor: Glimpse Sensor takes a full-sized image and a location, outputs the \emph{€retina-like}€ representation $\rho(x_t,l_{t-1})$ of the image $x_t$ around the given location $l_{t-1}$, which contains multiple resolution patches.
	\item Glimpse Network: takes as the inputs the retina representation $\rho(x_t,l_{t-1})$ and glimpse location $l_{t-1}$, and maps them into a hidden space using independent linear layers parameterized by $\theta_g^0$ and $\theta_g^1$ respectively using rectified units followed by another linear layer $\theta^2_g$ to combine the information from both components. Finally it outputs a glimpse representation $g_t$.
	\item Recurrent Neural Network (RNN): Overall, the model is an RNN. The core network takes as the input the glimpse representation $g_t$ at each step and history internal state $h_{t-1}$, then outputs a transition to a new state $h_{t}$, which is then mapped to action $a_t$ by an action network $f_a(\theta_a)$ and a new location $l_t$ by a location network $f_l(\theta_t)$. The location is to give an attention at next step, while the action, for image classification, gives a prediction based on current informations.  The prediction result, then, is used to generate the reward point, which is used to train these networks using Reinforcement Learning. 
	\item Loss Function: The reward could be based on classification accuracy (e.g. in Minh (2018) $r_t=1$ for correct classification and $r_t=0$ otherwise). In reinforcement learning, the loss can be finite-sum reward (in Minh (2018)) or discounted infinite reward. Cross-entropy loss for prediction at each time step is an alternative choice (which is not emphasized in the origin paper but added in the implementation given by Kevin below). It's also interested to figure out the difference function between these two loss and whether it's a good idea to use their combination. \end{itemize}

To evaluate your results, it is expected to see improved misclassification error compared against feedforward CNN without RAMs. Besides, you're encouraged to visualize the glimpse to see how the attention works for classification.

\subsection{More Reference}
\begin{itemize}
\item A PyTorch implementation of RAM by Kevin can be found here: 

\url{https://github.com/kevinzakka/recurrent-visual-attention}.
\item For reinforcement Learning, David Silver's course web could be found here 

\url{http://www0.cs.ucl.ac.uk/staff/d.silver/web/Teaching.html}

\noindent and Ruslan Satakhutdinov's course web at CMU is 

\url{http://www.cs.cmu.edu/~rsalakhu/10703/}
\end{itemize}



\section{Challenges from Project 2}
The following proposes three candidates and you are welcome to propose your own research project. Previous challenges are collected in the end and you may pursue a deeper exploration.

\subsection{Nexperia Kaggle in-class Contest}
Nexperia (\url{https://www.nexperia.com/}) is one of the biggest Semi-conductor company in the world. They will produce billions of semi-conductors every year. But unfortunately, they are facing a hard problem now which is the yield rate of the semi-conductors.
However they have lots of data and hope that the yield rate could be greatly improved by the hot deep learning technics now. So with the data they provide to us, we lunch this in-class Kaggle contest which tries to use various machine learning and deep learning methods to solve this real world problem.\\
Because this is the first Nexperia image classification contest, we set only 2 classes, one for bad semi-conductor and another for good. The aim of this simplified contest is to predict the type of each semi-conductor based on the image. And we provide 30K and 3000 images for training and testing respectively.\\

Checking the following kaggle website for more details. 
\begin{itemize}
	\item Kaggle: \url{https://www.kaggle.com/c/semi-conductor-image-classification-1}
\end{itemize}
\noindent To participate the contest, you need to login your Kaggle account first, then open the following invitation link and accept the Kaggle contest rule to download the data: 

{\center{\url{https://www.kaggle.com/t/7002fff75f2c422cb34068731971afcd}}}

\noindent In the future, we may consider to lunch another more complex in-class Kaggle contest. But this project is not limited to classification, you can do various explorations such as visualization \ref{fig:heatmap} and abnormal outlier detection.

\begin{figure}[htbp]
\begin{centering}
	\includegraphics[scale=0.5]{good.jpg}  
	\includegraphics[scale=0.5]{bad.jpg}  
	\caption{Sample Semi-conductor Image. Left: good Right: bad}
	\label{fig:nexperia}
	\end{centering}
\end{figure}
\begin{figure}[htbp]
\begin{centering}
	\includegraphics[scale=0.5]{heat.png}  
	\caption{Sample Semi-conductor Visualization.}
	\label{fig:heatmap}
\end{centering}
\end{figure}

\subsection{DCF-Net Exploration}
This challenge is to implement the DCF-Net in image classification tasks, defined by Xiuyuan Cheng et al. in the following paper

Qiang Qiu, Xiuyuan Cheng, Robert Calderbank, Guillermo Sapiro, \emph{DCFNet: Deep Neural Network with Decomposed Convolutional Filters}, ICML 2018. \url{arXiv:1802.04145.}

\noindent Currently there are two implementations, 
\begin{itemize}
\item Matlab:
\url{https://github.com/xycheng/DCFNet}
\item Pytorch:
\url{https://github.com/ZeWang95/DCFNet-Pytorch}
\end{itemize}

You may train a DCF-Net, e.g. DCF-Net-VGG16 or DCF-Net-ResNet18, on your favorite datasets (e.g. MNIST/Fashion-MNIST/Cifar10), and compare them against pretrained VGG16 and ResNet18 etc. For example, Table 4 of the DCF-Net paper, shows comparison between imagenet-vgg-verydeep-16 and DCFNet based VGG 16. You may either reproduce such an experiment and/or explore new models with new datasets. 

\subsection{Reproducible Training of CNNs} 

The following best award paper in ICLR 2017, 

\emph{Chiyuan Zhang, Samy Bengio, Moritz Hardt, Benjamin Recht, and Oriol Vinyals, Understanding deep learning requires rethinking generalization.} \url{https://arxiv.org/abs/1611.03530}

\noindent received lots of attention recently. Reproducibility is indispensable for good research. Can you reproduce some of their key experiments by yourself? The following are for examples. 

1. Achieve ZERO training error in standard and randomized experiments. As shown in Figure~\ref{fig:Recht1}, you need to train some CNNs (e.g. ResNet, over-parametric) with Cifar10 dataset, where the labels are true or randomly permuted, and the pixels are original or random (shuffled, noise, etc.), toward zero training error (misclassification error) as epochs grow. During the training, you might turn on and off various regularization methods to see the effects. If you use loss functions such as cross-entropy or hinge, you may also plots the training loss with respect to the epochs. 
\begin{figure}
\begin{centering}
\includegraphics[width=0.5\textwidth]{Recht1.png}  
\caption{Overparametric models achieve zero \emph{training error} (or near zero \emph{training loss}) as SGD epochs grow, in standard and randomized experiments.}
\label{fig:Recht1}
\end{centering}
\end{figure}

2. Non-overfitting of test error and overfitting of test loss when model complexity grows. Train several CNNs (ResNet) of different number of parameters, stop your SGD at certain large enough epochs (e.g. 1000) or zero \emph{training error (misclassification)} is reached. Then compare the \emph{test (validation) error} or \emph{test loss} as model complexity grows to see if you observe similar phenomenon in Figure~\ref{fig:Poggio1}: when \emph{training error} becomes zero, \emph{test error} (misclassification) does not overfit but \emph{test loss} (e.g. cross-entropy, exponential) shows overfitting as model complexity grows. This is for reproducing experiments in the following paper: 

\emph{Tomaso Poggio, K. Kawaguchi, Q. Liao, B. Miranda, L. Rosasco, X. Biox, J. Hidary, and H. Mhaskar. Theory of Deep Learning III: the non-overfitting puzzle}. Jan 30, 2018. \url{http://cbmm.mit.edu/publications/theory-deep-learning-iii-explaining-non-overfitting-puzzle} 

3. Can you give an analysis on what might be the reasons for the phenomena you observed? 

\begin{figure}
\begin{centering}
\includegraphics[width=0.9\textwidth]{Poggio1.png}  
\caption{When \emph{training error} becomes zero, \emph{test error} (misclassification) does not increase (resistance to overfitting) but \emph{test loss} (cross-entropy/hinge) increases showing overfitting as model complexity grows.}
\label{fig:Poggio1}
\end{centering}
\end{figure}


\section{Challenges from Project 1}
The basic challenge from Project 1 is 
\begin{itemize}
\item Feature extraction by scattering net with known invariants; 
\item Feature extraction by pre-trained deep neural networks, e.g. VGG19, and resnet18, etc.;
\item Visualize these features using classical unsupervised learning methods, e.g. PCA/MDS, Manifold Learning, t-SNE, etc.; 
\item Image classifications using traditional supervised learning methods based on the features extracted, e.g. LDA, logistic regression, SVM, random forests, etc.;
\item *Train the last layer or fine-tune the deep neural networks in your choice; 
\item Compare the results you obtained and give your own analysis on explaining the phenomena.
\end{itemize}

Below are two candidate datasets. Challenge marked by * above is only optional. 

\subsection{MNIST dataset}

Yann LeCun's website contains original MNIST dataset of 60,000 training images and 10,000 test images. 

\url{http://yann.lecun.com/exdb/mnist/}

There are various ways to download and parse MNIST files. For example, Python users may refer to the following website:

\url{https://github.com/datapythonista/mnist}

or MXNET tutorial on mnist

\url{https://mxnet.incubator.apache.org/tutorials/python/mnist.html}

\subsection{Fashion-MNIST dataset}

Zalando's Fashion-MNIST dataset of 60,000 training images and 10,000 test images, of size 28-by-28 in grayscale. 

\url{https://github.com/zalandoresearch/fashion-mnist}

As a reference, here is Jason Wu, Peng Xu, and Nayeon Lee's exploration on the dataset in project 1:

\url{https://deeplearning-math.github.io/slides/Project1_WuXuLee.pdf}

\subsection{Cifar10 dataset} 

The Cifar10 dataset consists of 60,000 color images of size 32x32x3 in 10 classes, with 6000 images per class. It can be found at 

\url{https://www.cs.toronto.edu/~kriz/cifar.html}

%\noindent Attention: training CNNs with such a dataset is time-consuming, so GPU is usually adopted. If you would like an easier dataset without GPUs, perhaps use MNIST or Fashion-MNIST (introduced below).


\subsection{Identification of Raphael's paintings from the forgeries}

The following data, provided by Prof. Yang WANG from HKUST,

\url{https://drive.google.com/folderview?id=0B-yDtwSjhaSCZ2FqN3AxQ3NJNTA&usp=sharing}

\noindent contains a 28 digital paintings of Raphael or forgeries. Note that there are both jpeg and tiff files, so be careful with the bit depth in digitization. The following file

\url{https://docs.google.com/document/d/1tMaaSIrYwNFZZ2cEJdx1DfFscIfERd5Dp2U7K1ekjTI/edit}

\noindent contains the labels of such paintings, which are 
\begin{enumerate}
\item[1] Maybe Raphael - Disputed
\item[2] Raphael
\item[3] Raphael
\item[4] Raphael
\item[5] Raphael
\item[6] Raphael
\item[7] Maybe Raphael - Disputed
\item[8] Raphael
\item[9] Raphael
\item[10] Maybe Raphael - Disputed
\item[11] Not Raphael
\item[12] Not Raphael
\item[13] Not Raphael
\item[14] Not Raphael
\item[15] Not Raphael
\item[16] Not Raphael
\item[17] Not Raphael
\item[18] Not Raphael
\item[19] Not Raphael
\item[20] My Drawing (Raphael?)
\item[21] Raphael
\item[22] Raphael
\item[23] Maybe Raphael - Disputed
\item[24] Raphael
\item[25] Maybe Raphael - Disputed
\item[26] Maybe Raphael - Disputed
\item[27] Raphael
\item[28] Raphael
\end{enumerate}
There are some pictures whose names are ended with alphabet like A's, which are irrelevant for the project. 

The challenge of Raphael dataset is: can you exploit the known Raphael vs. Not Raphael data to predict the identity of those 6 disputed paintings (maybe Raphael)? Textures in these drawings may disclose the behaviour movements of artist in his work. One preliminary study in this project can be: \emph{take all the known Raphael and Non-Raphael drawings and use leave-one-out test to predict the identity of the left out image; you may break the images into many small patches and use the known identity as its class.}      

The following student poster reports are good explorations

1) Hanlin GU, Yifei HUANG, and Jiaze SUN: \url{https://github.com/yuany-pku/2017_CSIC5011/blob/master/project3/05.GuHuangSun_poster.pdf}
%\url{http://math.stanford.edu/~yuany/course/2015.fall/poster/Raphael_LI\%2CYue_1300010601.pdf}

2) Jianhui ZHANG, Hongming ZHANG, Weizhi ZHU, and Min FAN: \url{https://deeplearning-math.github.io/slides/Project1_ZhangZhangZhuFan.pdf},

3) Wei HU, Yuqi ZHAO, Rougang YE, and Ruijian HAN: \url{https://deeplearning-math.github.io/slides/Project1_HuZhaoYeHan.pdf}.


The following papers by Haixia Liu et al. study art authentication using geometric tight frames and scattering transform, respectively, which might be useful reference for you:

\url{http://dx.doi.org/10.1016/j.acha.2015.11.005}

\url{https://www.sciencedirect.com/science/article/pii/S0165168418301105}


%\section{Air Quality Weibo Data} (courtesy of Prof. Xiaojin Zhu from University of Wisconsin at Madison) 
%You can login my server:
%
%\texttt{ssh einstein@162.105.205.92}
%
%\noindent using the password I provided on class. 
%
%On the read-only folder \texttt{/data/AQweibo/}, the \texttt{AQICityData/} directory contains the Weibo posts, the AQI for 108 cities with (AQI) information during the study period
%from 2013-11-18 to 2013-12-18 (both inclusive); Information for the spatiotempral bin (city,date) is in the directory \texttt{city\_date/}. See \texttt{README.txt} for more information.
%
%


%\section{Raph}
%The following data contains 1258-by-452 matrix with closed prices of 452 stocks in SNP'500 for workdays in 4 years.
%
%\url{http://math.stanford.edu/~yuany/course/data/snp452-data.mat} 
%
%\noindent or in R: 
%
%\url{http://math.stanford.edu/~yuany/course/data/snp500.Rda}
%
%%You may use PCA to explore the `invisible hands' of markets.
%
%\section{Animal Sleeping Data} The following data contains animal sleeping hours together with other features: 
%
%\url{http://math.stanford.edu/~yuany/course/data/sleep1.csv}
%
%
%\section{US Crime Data} The following data contains crime rates in 59 US cities during 1970-1992:
%
%\url{http://math.stanford.edu/~yuany/course/data/crime.zip}
%
%\noindent Some students in previous classes study crime prediction in comparison with MLE and James-Stein, for example, see
%
%\url{https://github.com/yuany-pku/2017_math6380/blob/master/project1/DongLoXia_slides.pptx}
%
%
%\section{NIPS paper datasets}
%NIPS is one of the major machine learning conferences. The following datasets collect NIPS papers:
%
%\subsection{NIPS papers (1987-2016)} The following website: 
%
%\url{https://www.kaggle.com/benhamner/nips-papers}
%
%\noindent collects titles, authors, abstracts, and extracted text for all NIPS papers during 1987-2016. In particular the file {\texttt{paper\_authors.csv}} contains a sparse matrix of paper coauthors. 
%
%\subsection{NIPS words (1987-2015)} The following website:
%
%\url{https://archive.ics.uci.edu/ml/datasets/NIPS+Conference+Papers+1987-2015}
%
%\noindent collects the distribution of words in the full text of the NIPS conference papers published from 1987 to 2015. The dataset is in the form of a 11463 x 5812 matrix of word counts, containing 11463 words and 5811 NIPS conference papers (the first column contains the list of words). Each column contains the number of times each word appears in the corresponding document. The names of the columns give information about each document and its timestamp in the following format: {\texttt{Xyear\_paperID}}. 
%
%
%\section{Jiashun Jin's data on Coauthorship and Citation Networks for Statisticians}
%Thanks to Prof. Jiashun Jin at CMU, who provides his collection of citation and coauthor data for statisticians. The data set covers all papers between 2003 and the first quarter of 2012 from the Annals of Statistics, Journal of the American Statistical Association, Biometrika and Journal of the Royal Statistical Society Series B. The paper corrections and errata are not included. There are 3607 authors and 3248 papers in total. The zipped data file (14M) can be found at 
%
%\url{http://math.stanford.edu/~yuany/course/data/jiashun/Jiashun.zip}
%
%\noindent with an explanation file
%
%\url{http://math.stanford.edu/~yuany/course/data/jiashun/ReadMe.txt}
%
%With the aid of Mr. LI, Xiao, a subset consisting 35 COPSS award winners (\url{https://en.wikipedia.org/wiki/COPSS_Presidents\%27_Award}) up to 2015, is contained in the following file
%
%\url{http://math.stanford.edu/~yuany/course/data/copss.txt} 
%
%\noindent An example was given in the following article, A Tutorial of Libra: R Package of Linearized Bregman Algorithms in High Dimensional Statistics, downloaded at
%
%\url{http://math.stanford.edu/~yuany/course/reference/Libra_Tutorial_springer.pdf}
%
%The citation of this dataset is: \emph{P. Ji and J. Jin. Coauthorship and citation networks for statisticians. Ann. Appl. Stat. Volume 10, Number 4 (2016), 1779-1812}, (\url{http://projecteuclid.org/current/euclid.aoas})
%
%
%
%
%\section{Co-appearance data in novels: Dream of Red Mansion and Journey to the West}
%
%A 374-by-475 binary matrix of character-event can be found at the course website, in .XLS, .CSV, .RData, and .MAT formats. For example the RData format is found at
%
%\url{http://math.stanford.edu/~yuany/course/data/dream.RData} 
%
%\noindent with a readme file:
%
%\url{http://math.stanford.edu/~yuany/course/data/dream.Rd}
%
%\noindent as well as the .txt file which is readable by R command {\tt read.table()},
%
%\url{http://math.stanford.edu/~yuany/course/data/HongLouMeng374.txt}
%
%\url{http://math.stanford.edu/~yuany/course/data/readme.m}
%
%Thanks to Ms. WAN, Mengting, who helps clean the data and kindly shares her BS thesis for your reference
% 
%\url{http://math.stanford.edu/~yuany/report/WANMengTing2013_HLM.pdf}
%
%%Among various choices of analysis, with this data matrix $X$, you may form a weighted graph $W=X * X'$, pursue PCA of $X$, and sparse SVD of $X$ etc. As an example, here is a project presentation by LI, Liying which gives an analysis of A Journey to the West (by Chen-En Wu) based on PCA, for the class Mathematical Introduction to Data Science in Fall 2012 where you may find more interesting approaches.
%%
%%\url{http://www.math.pku.edu.cn/teachers/yaoy/reference/LiyingLI_Xiyouji2012_slides.pdf}
%
%Moreover you may find a similar matrix of 302-by-408 for the Journey to the West (by Chen-En Wu) at:
%
%\url{http://math.stanford.edu/~yuany/course/data/west.RData}
%
%\noindent whose matlab format is saved at
%
%\url{http://math.stanford.edu/~yuany/course/data/xiyouji.mat}

%%%%%%


%\section{Drug Efficacy Data}
%
%Thanks to Prof. Xianting Ding at Shanghai Jiao Tong University and Prof. Chih-Ming Ho from University of California at Los Angeles, we have the following datasets on combinatorial drug efficacy.
%
%The first dataset consists of two experiments, all with the same 4 drugs in cell lines for attacking leukemia, with 256 experiments of combinatorial drug dosage at 4 levels. The response is the therapeutic window measuring the efficacy with a trade-off by toxicity. 
% 
% \url{http://math.stanford.edu/~yuany/course/data/Ding_4drugs.xlsx}
%
%\noindent whose drugs are explained in 
%
%\url{http://math.stanford.edu/~yuany/course/data/Ding_4drugs_readme.pdf}
%
%Can you find a good prediction of drug response efficacy using those combinatorial dosage levels? It was suggested that quadratic polynomials at logarithmic dosage levels are good models in personalized medicine, e.g. the following cover paper in Science \emph{Translation Medicine}:
%
%\url{http://stm.sciencemag.org/content/8/333/333ra49}
%
%\noindent with a sample 14 drug efficacy at level 2 experiment data in liver transplant: 
%
%\url{http://math.stanford.edu/yuany/course/data/TB-FSC-03A-data.xlsx}

%\section{Drug Sensitivity Data by Cleave}
%The following dataset is kindly provided by Cleave Co. Ltd. USA, for the exploration on class. {\textbf{Please keep its use only in this class and any publication will be subject to the approval of Cleave.}}
%
%The dataset is contained in the following zip file (73M).
%
%\url{http://math.stanford.edu/~yuany/course/data/cleave.zip}
%
%\noindent where you may find
%\begin{enumerate}
%\item \texttt{data explanation.pptx}: description of data in pptx
%\item \texttt{data for Yuan Yao.xlsx}: data file
%\item \texttt{Gene set collection 1 for Yuan Yao.txt}: gene set collection
%\item \texttt{Gene set collection 2 for Yuan Yao.txt}: gene set collection
%\item \texttt{reference}: a folder contains a survey paper on 40+ machine learning algorithms as well as some source codes -- \emph{Nature Biotechnology 32, 1202--1212 (2014)} (\url{http://www.nature.com/nbt/journal/v32/n12/full/nbt.2877.html})
%\end{enumerate}
%
%The basic problem is to predict the drug response \texttt{IC50 within 72 hours}, using all the information collected so far, introduced by Ms. Lijing Wang with slides
%
%\url{http://math.stanford.edu/~yuany/course/2016.spring/cleave_lijing.pdf}
%
%\noindent as well as our CPH'2017 poster
%
%\url{http://math.stanford.edu/~yuany/publications/poster_CleaveBioCPH2017_ForReview.pdf}
%
%\noindent where the crucial discovery is that recursive variable selection by LASSO is more effective than one-stage LASSO. 

%\subsection{The Characters in A Dream of Red Mansion} 
%
%A 376-by-475 matrix of character-event can be found at the course website, in .XLS, .CSV, and .MAT formats. For example the Matlab format is found at
%
%\url{http://www.math.pku.edu.cn/teachers/yaoy/data/hongloumeng/hongloumeng376.mat} 
%
%\noindent with a readme file:
%
%\url{http://www.math.pku.edu.cn/teachers/yaoy/data/hongloumeng/readme.m}
%
%Thanks to Ms. WAN, Mengting (now at UIUC), an update of data matrix consisting 374 characters (two of 376 are repeated) which is readable by R read.table() can be found at 
%
%\url{http://www.math.pku.edu.cn/teachers/yaoy/data/hongloumeng/HongLouMeng374.txt}
%
%\noindent She also kindly shares her BS thesis for your reference
% 
% \url{http://www.math.pku.edu.cn/teachers/yaoy/reference/WANMengTing2013_HLM.pdf}
%
%% Among various choices of analysis, with this data matrix $X$, you may form a weighted graph $W=X * X'$, pursue PCA of $X$. 
%
%\subsection{A Journal to the West} On course website, you may also find the link to this dataset with a 302-by-408 matrix, whose matlab format is saved at
%
%\url{http://www.math.pku.edu.cn/teachers/yaoy/Fall2011/xiyouji/xiyouji.mat}
%
%For your reference, here is a project presentation by Mr. LI, Liying (at PKU) which gives an analysis based on PCA
%
%\url{http://www.math.pku.edu.cn/teachers/yaoy/reference/LiyingLI_Xiyouji2012_slides.pdf}
%

%\section{Heart PCI Operation Effect Prediction}
%
%The following data, provided by Dr. Jinwen Wang at Anzhen Hospital, 
%
%\url{http://math.stanford.edu/~yuany/course/data/heartData_20140401.xlsx}
%
%\noindent contains 2581 patients with 73 measurements (inputs) as well as a response variable indicating if after the heart operation there is a null-reflux state. This is a classification problem, with a challenge from the large amount of missing values. Sheet 3 and 4 in the file contains some explanation of the data and variables. 
%
%The problems are listed here:
%\begin{enumerate}
%\item The inputs (covariates) are of three kinds, measurements upon check-in, measurements before PCI operation, and measurements in PCI operations. For doctors, it is desired to find a prediction model based on measurements before the operation (including check-in). Sheet 2 in the file contains only such measurements.
%\subitem The following two reports by LV, Yuan and LI, Xiao, respectively, might be interesting to you:
%
%\url{http://math.stanford.edu/~yuany/course/reference/MSThesis.LvYuan.pdf} 
%
%\url{http://arxiv.org/abs/1511.04656} 
%
%\item It is also an interesting problem how to predict the effect based on all measurements, with lots of missing values. Sheet 1 contains the full measurements. There are some good work by previous students, which are listed here for your reference: 
%%\subitem The following two reports by LU, Yu and WANG, Qing, are probably inspiring to you.
%%
%%\url{http://www.math.pku.edu.cn/teachers/yaoy/reference/LuYu_201303_BigHeart.pdf} 
%%
%%\url{http://www.math.pku.edu.cn/teachers/yaoy/reference/WangQing_201303_BigHeart.pdf} 
%
%\subitem The following report by MIAO, Wang and LI, Yanfang, pioneers in missing value treatment. 
%
%\url{http://math.stanford.edu/~yuany/course/reference/MiaoLi2013S_project01.pdf}
%
%\end{enumerate} 

%\emph{In the final project, it is desired to take only those measurements upon check-in to predict the probability of non-reflux (non-reflow) after PCI operations. An interpretable model adds a big value! You may compare with your first warm-up project to show your improvements.} 
%
%
%\section{SNPs Data}
% This dataset contains a data matrix $X\in \R^{p\times n}$ of about $n=650,000$ columns of SNPs (Single Nucleid Polymorphisms) and $p=1064$ rows of peoples around the world. Each element is of three choices, $0$ (for `AA'), $1$ (for `AC'), $2$ (for `CC'), and some missing values marked by $9$. 
%
%\url{http://math.stanford.edu/~yuany/course/ceph_hgdp_minor_code_XNA.txt.zip}
%
%\noindent which is big (151MB in zip and 2GB original txt). Moreover, the following file contains the region where each people comes from, as well as two variables {\texttt{ind1}} and{\texttt{ind2}} such that $X({\texttt{ind1}},{\texttt{ind2}})$ removes all missing values. 
%
%\url{http://math.stanford.edu/~yuany/course/data/HGDP_region.mat}
%
%\noindent More detailed information about these persons in the dataset can be also found at
%
%\url{http://math.stanford.edu/~yuany/course/data/HGDPid_populations_ALL.xls}
%
%Some results by PCA can be found in the following paper, Supplementary Information. 
%
%\url{http://www.sciencemag.org/content/319/5866/1100.abstract}
%
%\section{Protein Folding} 
%Consider the 3D structure reconstruction based on incomplete MDS with uncertainty. Data file: 
%
%\url{http://math.stanford.edu/~yuany/course/data/protein3D.zip}
%
%\begin{figure}[htbp]
%\begin{center}
%\includegraphics[width=0.5\textwidth]{../2013_Spring_PKU/Yes_Human.png}  
%\caption{3D graphs of file PF00018\_2HDA.pdf (YES\_HUMAN/97-144, PDB 2HDA)}
%\label{yes_human}
%\end{center}
%\end{figure}
%
%\noindent In the file, you will find 3D coordinates for the following three protein families: 
%\subitem PF00013 (PCBP1\_HUMAN/281-343, PDB 1WVN), \\
%\subitem PF00018 (YES\_HUMAN/97-144, PDB 2HDA), and \\
%\subitem PF00254 (O45418\_CAEEL/24-118, PDB 1R9H). \\
%
%For example, the file {\tt PF00018\_2HDA.pdb} contains the 3D coordinates of alpha-carbons for a particular amino acid sequence in the family, YES\_HUMAN/97-144, read as
%
%{\tt{VALYDYEARTTEDLSFKKGERFQIINNTEGDWWEARSIATGKNGYIPS}}
%
%\noindent where the first line in the file is 
%
%97	V	0.967	18.470	4.342
%
%\noindent Here
%\begin{itemize}
%\item `97': start position 97 in the sequence
%\item `V': first character in the sequence
%\item $[x,y,z]$: 3D coordinates in unit $\AA$.
%\end{itemize}
%
%\noindent Figure \ref{yes_human} gives a 3D representation of its structure. 
%
%
%Given the 3D coordinates of the amino acids in the sequence, one can computer pairwise distance between amino acids, $[d_{ij}]^{l\times l}$ where $l$ is the sequence length. A \emph{contact map} is defined to be a graph $G_\theta=(V,E)$ consisting $l$ vertices for amino acids such that and edge $(i,j)\in E$ if $d_{ij} \leq \theta$, where the threshold is typically $\theta=5\AA$ or $8\AA$ here. 
%
%Can you recover the 3D structure of such proteins, up to an Euclidean transformation (rotation and translation), given noisy pairwise distances restricted on the contact map graph $G_\theta$, i.e. given noisy pairwise distances between vertex pairs whose true distances are no more than $\theta$? Design a noise model (e.g. Gaussian or uniformly bounded) for your experiments. 
%
%When $\theta=\infty$ without noise, classical MDS will work; but for a finite $\theta$ with noisy measurements, SDP approach can be useful. You may try the matlab package SNLSDP by Kim-Chuan Toh, Pratik Biswas, and Yinyu Ye, downladable at \url{http://www.math.nus.edu.sg/~mattohkc/SNLSDP.html}. 
%


%Attention: this last dataset is relatively big with about 2GB size. 
%
%You can login my server:
%
%\texttt{ssh einstein@162.105.205.92}
%
%\noindent using the password I provided on class. On the read only folder \texttt{/data/snp/}, you will find all the data in both .txt and .mat (\texttt{data.mat, HGDP\_region.mat, readme.m}).



%\subsection{Bird Flu Dataset} (courtesy of Steve Smale and Cissy) This dataset 162 H5N1 (bird flu) virus sequences discovered around the world:
%
%\url{http://www.math.pku.edu.cn/teachers/yaoy/data/birdflu_seq162.txt} 
%
%Locations of such virus discovered are reported with latitude and longitude coordinates on the globe:
%
%\url{http://www.math.pku.edu.cn/teachers/yaoy/data/birdflu_latgrat.txt} 
%
%Pairwise geodesic distances between these 162 sites are constructed as  
%
%\url{http://www.math.pku.edu.cn/teachers/yaoy/data/birdflu_geodist.txt}
%
%A kernel-induced $l_2$-distances between 162 virus sequences are given in 
%
%\url{http://www.math.pku.edu.cn/teachers/yaoy/data/birdflu_l2dist.txt}
\end{document}


